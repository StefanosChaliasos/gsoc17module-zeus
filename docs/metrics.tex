\documentclass{article}

\usepackage{url}
\usepackage{graphicx}
\usepackage{listings}
\usepackage{todonotes}
\usepackage{algorithm}
\usepackage{algpseudocode}

\usepackage{minted}

\presetkeys{todonotes}{fancyline, color=yellow!30}{}

\date{}
\begin{document}

\title{Metrics for the AsiaCrypt16 Implementation} 

\author{Vitalis Salis}

\maketitle
\begin{abstract}
  We have computed some metrics for the prototype implementation
  \cite{prototype} of a mixnet based on the shuffle argument proposed
  by Fauzi et al \cite{shufflearg}. The goal of these metrics is to
  identify aspects of the code that are slow and find suitable
  replacements for them.
\end{abstract}

\section{Introduction}

The prototype implementation of the mixnet proposed by Fauzi et al,
produces multiple implementation difficulties. On implementations of
cryptographic protocols it is typical to use C for your cryptographic
computations. Yet the prototype is implemented using python, so it has
to switch between python and C for its operations. This may be a
bottleneck of the prototype and the reason some operations are slower
than they should. Another reason may be that the underlying C
cryptographic operations themselves are not efficient, and a different
C implementation might improve matters. The two reasons are not
exclusive, and one might compound the other.

\section{Metrics}

Table 1 contains a list of metrics for the various operations of the
prototype. Most of the time is taken by the prover and the verifier,
as expected, because these have the most computations that produce a
context switch between Python and C.

\begin{table}
\begin{tabular}{ |p{3cm}|p{5cm}|p{3cm}|  }
    \hline
    \multicolumn{3}{|c|}{Metrics}\\
    \hline
    Operation & Short Description & Time per 100 voters\\
    \hline
    Initialization & Creates the elliptic Curve and private keys & 364ms\\
    Encryption & Encrypts the votes & 674ms\\
    Random Permutations & Creates random numbers & 1ms\\
    Proof & The shuffle & 2085ms\\
    Verification & Verification of the shuffle & 2738ms\\
    Decryption & Decrypts the votes & 489ms\\
    \hline
\end{tabular}
\caption{Metrics}
\end{table}

The time taken by each of these operations is linear, meaning that for
200 ciphertexts the numbers on the table are doubled.

\section{Context Switches}

A context switch happens when a python program communicates with a C
program for various cryptographic computations. The reasoning behind
believing that a context switch may be the bottleneck of the application
is that python needs to create a PyObject containing the
data it wants to communicate, and C also needs to create a PyObject to
return the result of the computations.

The prover has various steps. In order to validate our theory about
context switches we measured each of these steps. Two of those steps,
while having the same number of iterations, had a significant time
difference. In particular, \mintinline{python}{step2a} below took
100ms, while \mintinline{python}{step3a} took 700ms.

\begin{minted}[breaklines]{python}
def step2a(sigma, A1, randoms, g1_poly_zero, g1rho, g1_poly_squares):
    pi_1sp = []
    inverted_sigma = inverse_perm(sigma)
    for inv_i, ri, Ai1 in zip(inverted_sigma, randoms, A1):
        g1i_poly_sq = g1_poly_squares[inv_i]
        v = (2 * ri) * (Ai1 + g1_poly_zero) - (ri * ri) * g1rho + g1i_poly_sq
        pi_1sp.append(v)
    return pi_1sp
  \end{minted}

\begin{minted}[breaklines]{python}
def step3a(sigma, ciphertexts, s_randoms, pk1, pk2):
    v1s_prime = []
    v2s_prime = []
    for perm_i, s_random in zip(sigma, s_randoms):
        (v1, v2) = ciphertexts[perm_i]
        v1s_prime.append(tuple_add(v1, enc(pk1, s_random[0], s_random[1], 0)))
        v2s_prime.append(tuple_add(v2, enc(pk2, s_random[0], s_random[1], 0)))
    return list(zip(v1s_prime, v2s_prime))
\end{minted}
  
\noindent
First we attributed the time difference to various calls to zip and to tuple
creation. After removing all the calls to zip we didn't notice any significant
difference. This seemed to validate the context switches theory, because the
slower step contained more context switches per iteration.

But that's not the case. Using cProfile we identified the main
reason behind this difference. The slower step does more multiplications on
elliptic curve elements. While it is expected that multiplication will be slower
than addition, the difference was enough to dismiss the context switches theory.

Multiplication on our elliptic curve elements takes 575ms per 300
multiplications, while addition takes 5ms for 400 additions. If the real
problem were context switches, then the addition wouldn't have such a huge
difference with the multiplication, because it has more operations hence
more context switches.


\section{Comparing bplib and libsnark}

The prototype implementation uses the bplib\cite{bplib} python module.
bplib implements bilinear pairings on elliptic curves while also supporting
elliptic curve operations using the openssl library.

Another implementation supporting elliptic curve computations and bilinear
pairings is libsnark\cite{libsnark}.

The common characteristics of these libraries are that they both use the
Ate Pairing and they use windowed exponentiation for
optimization purposes.

A key difference of these implementations is that bplib uses the
curve Fp254BNb, while libsnark uses bn128 which is a patch on the
Fp254BNb curve. Also, libsnark supports vectorized exponentiation
which boosts up its performance.

In order to compare these two libraries and validate our theory that
libsnark is faster than bplib, we created two different tests
using bplib and libsnark on each one. The tests did multiplications
(the bottleneck of the prototype) on both elliptic curve groups.

The results validated our theory. Multiplying elements on the G2
group using libsnark yielded a performance of 0.38s/1000 ciphertexts
while using openssl yielded 3.22s/1000. On G1 libsnark produced
0.13s/1000 while bplib produced 0.96s/1000 multiplications.


\section{Wrapping libsnark with Cython}

Since libsnark computes multiplications faster than the openssl
implementation, the most obvious solution is to replace openssl
with libsnark. The best candidate for this job is Cython, because
it offers the performance aspects of C, while providing the functionality
of Python. In order to validate that Cython, indeed will yield better
performance we created a basic Cython application that does multiplications
on the G2 group of the libsnark elliptic curve.

The results were positive. A multiplication on the G2 group using Cython
takes about 0.5ms while on our prototype implementation, that uses bplib,
a multiplication takes about 2ms. So that's a 4x boost in performance.

Also our Cython implementation didn't implement vectorized multiplications,
so there's still room for optimizations.


\section{Solutions}

Since the real bottleneck are the multiplications on G2 elements,
the most obvious solution is to use optimizations on the multiplication
process.

As mentioned, libsnark computes multiplications faster than
our current implementation. Yet libsnark is written in C++
and we want to use a python module. A python wrapper for
libsnark would be useful, for our needs and the open source community.


\bibliographystyle{plain}
\bibliography{metrics}

\end{document} 
